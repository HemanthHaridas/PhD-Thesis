\chapter[Conclusions and Future Directions]{Conclusions and Future Directions}
\justifying
This thesis focuses on the development and application of Drude polarizable force field (FF) parameters to model graphene surfaces and their interactions with nucleobases and single-stranded DNA (ssDNA). We found that polarizable FFs, despite being more computationally demanding, provide a more accurate representation of the system's molecular polarizability compared to non-polarizable additive FFs. This is particularly important for accurately modeling the non-covalent interactions between nucleobases and graphene, which are key to understanding the nanopore clogging observed in ssDNA translocation experiments.

Chapter 3 details the development of Drude polarizable FF parameters for graphene. These parameters were derived from first-principles calculations and validated against nucleobase-graphene binding energy curves from ab-initio calculations and isothermal titration calorimetry (ITC) experiments.

In Chapter 4, building on the insights from Chapter 3, we used polarizable FF simulations to study how nucleobases self-assemble on a graphene surface at varying concentrations. Our simulations accurately captured the formation of two-dimensional (2D) structures, consistent with scanning tunneling microscopy (STM) images of cytosine nucleobases on highly oriented pyrolytic graphite (HOPG) surfaces.

Chapter 5 extends the use of polarizable FF simulations to study the dynamics of electrolyte solutions near a polarizable graphene surface. We were able to accurately reproduce ion-graphene interactions and observed a shift in the binding pattern of Cl\textsuperscript{-} ions, in line with ab-initio calculations and deep-ultraviolet experiments of SCN\textsuperscript{-} ions binding on graphene.

Finally, in Chapter 7, we applied the knowledge gained from the previous chapters to investigate the translocation dynamics of a ssDNA homopolymer through a graphene nanopore. Our focus was to understand the causes of irreversible pore blocking observed in ssDNA translocation experiments. We found that the formation of a knot-like structure in the dC\textsubscript{20} strand, stabilized by extensive intermolecular hydrogen bonding and $\pi$-$\pi$ stacking interactions, is a significant contributor to the observed pore blockade. This finding contrasts with previous studies using non-polarizable additive FF simulations, which reported clean translocation through the nanopore. Our study underscores the importance of including molecular polarizability in simulations, as our results closely align with experimental findings.

This thesis provides a framework for studying the self-assembly of small molecules and the translocation dynamics of single-stranded DNA (ssDNA). Future research could extend this work in several interesting ways:
\begin{enumerate}
    \item \textbf{Investigate Signal-Dependent Self-Assemblies:} Explore how the formation of self-assemblies can be controlled by chemical or electrical signals. This could involve studying how different self-assemblies form based on the presence or absence of a signal. For instance, one could examine the formation of ribbon-like structures and G4-quadruplexes in guanosine, depending on whether free K+ ions are present in the solution.
    \item \textbf{Extend Parameterization to Other 2D Materials:} It would be beneficial to extend the parameterization to other two-dimensional materials of interest, such as hexagonal boron nitride (\textit{h}-BN). This would pave the way for investigating van der Waals heterostructures using CHARMM force fields. One potential area of investigation is the variation in ssDNA translocation dynamics through sandwich nanopores made from graphene and \textit{h}-BN sheets.
    \item \textbf{Study ssDNA Heteropolymers:} In Chapter 6, our simulations used ssDNA homopolymers to provide a proof-of-concept for further investigations. It would be worthwhile to extend this study to ssDNA heteropolymers, bringing the simulated system closer to those studied in experiments. An intriguing avenue of exploration could be the application of a gradually increasing voltage, as opposed to the current step-wise increases in applied voltage. This could provide more insights into the translocation dynamics of ssDNA.
\end{enumerate}
